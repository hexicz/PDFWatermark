\chapter{Úvod}
V této kapitole bude popsána motivace pro vznik portálu pro zabezpečenou distribuci PDF\footnote{Portable Document Format} dokumentů.

\section{Motivace}
V dnešní době, kdy je nedostatek kvalitních studijních materiálů, je potřeba podporovat vznik těchto materiálů. U vyučujících se projevuje nevole k tvorbě kvalitních elektronických studijních materiálů, a to především z důvodu téměř nemožné kontroly nad jejich distribucí. Potencionální autoři kvalitních studijních materiálů nechtějí investovat svůj volný čas do psaní s tím, že se jejich díla budou nekontrolovatelně šířít po celém Internetu. Tradiční metoda distribuce studijních materiálů, formou skriptm nemusí být v mnoha případech řešením. V některých oborech, zejména technických, se standardy mění tak rychle, že není dost dobře možné vydávat dostatečně rychle aktualizovaná skripta vyhovující současným trendům.

V okamžiku, kdy by měli vyučující možnost dostatečné kontroly nad šířením svých studijních materiálů, mohli by rozšířit spektrum kvalitních volně dostupných studijních materiálů ke svým kurzům. Tyto materiály by byly dostupné pouze studentům ČVUT ke kvalitní příprave ke studiu. Současný stav nutí studenty spoléhat se pouze na nedostatečné materiály ve formě prezentací vytvořených za účelem přednášek a cvičení. V těchto prezentacích se nevyskytují zdaleka všechny důležité informace potřebné k úspěšnému zakončení kurzu. Přesto, že by si studenti měli dělat z přednášek a cvičení vlastní poznámky ke studiu, může se stát, že se studenti z různých důvodů (např. nemoci) nemohou dostavit na přednášku a pořídit si důležité poznámky, bez kterých nebudou schopni kurz úspěšně dokončit.

Další možností motivace vyučujících k vytváření kvalitních studijních materiálů by byla možnost zpřístupnit tyto materiály studentům ČVUT za určitý finanční obnos. Pokud budou tyto kvalitní studijní materiály za rozumnou a pro studenty přijatelnou cenu lehce přístupné, sníží se potřeba studentů shánět nelegální, většinou i nekvalitní a nevhodné materiály z různých neověřených zdrojů.

Z výše uvedeného vyplývá, že vytvoření kvalitního portálu pro zabezpečenou distribuci elektronických studijních materiálů by mohlo zvýšit chuť vyučujících k tvorby kvalitních studijních materiálů. Zároveň by vznik takovéhoto portálu mohl zlepšit kvalitu přípravy studentů na úspěšné absolvování studia na vysoké škole.  

\section{Historie}
Tato bakalářská práce rozšiřuje již existující polofunkční prototyp vyvinut na fakultě v rámci studentských prací. Tento prototyp, umožňuje vložit do PDF dokumentů vodoznak s informacemi na základě vyplnění formuláře. Zároveň je možné tyto dokumenty volně stahovat pomocí vygenerovaného obsahu.
