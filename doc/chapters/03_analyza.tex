\chapter{Analýza}

\section{Uživatelé}
Předpokládá se, že s aplikací budou pracovat různí uživatelé z akademického prostředí ČVUT FEL. Činnosti těchto uživatelů půjdou rozdělit do dvou hlavních skupin. 

První skupinou budou uživatelé, kteří budou aplikaci zásobovat obsahem. Tedy hlavně vyučující a další zaměstnanci školy. Jejich primární činností bude nahrát vybraný PDF dokument do aplikace, nastavit jeho vlastnosti a sdílet ho pomocí systémem vygenerovaných odkazů. Mezi další činnosti bude patřit editace vlastností PDF dokumentů, případně nahrazení dokumentu aktuální verzí a v neposlední řadě také mazání dokumentů.

Nejpočetnější skupina uživatelů aplikace bude tvořena studenty, kteří budou na základě získaných odkazů stahovat požadované materiály ve formě PDF dokumentů. 

\subsection{Uživatelské role}
Na základě analýzy uživatelů systému byly navrhnuty uživatelské role. Pro jednoduchost jsou role navrženy tak, aby je bylo možné generalizovat. Generalizace uživatelských rolí znamená, že uživatel s rolí vyššího stupně má zároveň vlastnosti uživatelů s rolemi nižších stupňů. Uživatelské role byly zvoleny takto:

\begin{itemize}
    \item \textbf{Host} je jakýkoliv uživatel, který vstoupí do aplikace a nepřihlásí se. Takový uživatel nemá prakticky žádné práva v aplikaci a bude mu zobrazena pouze úvodní stránka. Při pokusu vykonání jakékoliv aktivity v systému bude vyzván k přihlášení přes FELid.
    \item \textbf{Student} je přihlášený uživatel, který je v systému FELid evidován jako student. Student má právo stáhnout si pomocí odkazu PDF dokument, který bude před stažením označen vodotiskem se jménem studenta a datem stažení dokumentu.  
    \item \textbf{Vyučující} je také přihlášený uživatel, který je v systému FELid evidován jako zaměstnanec fakulty. Vyučující má stejné právo stahovat PDF dokumenty jako Student a zároveň může do aplikace vkládat a následně i spravovat nové dokumenty. Každý uživatel s rolí Vyučující bude mít právo spravovat pouze PDF dokumenty, které sám do systému vložil.
    \item \textbf{Administrátor} bude přihlášený uživatel, který bude mít právo spravovat aplikaci. Administrátor bude mít stejná prává jako Vyučují s tím, že narozdíl od něj bude mít přístup ke správě všech dokumentů od všech možných Vyučujících.
\end{itemize}

%\section{Doménový model}
%Doménový model bude velmi jednoduchý. Bude obsahovat jen několik málo entit. V této sekci jsou popsány použité entity a jejich atributy.
%
%\begin{itemize}
%    \item 
%\end{itemize}

\section{Požadavky}
V této sekci jsou uvedeny požadavky na portál z pohledu zadání práce.

\subsection{Funkční požadavky}
Funkční požadavky nám říkají, jaké akce bude systém umožňovat uživatelům provádět. Tyto akce jsou pokryty příůady užití. \cite{UMLBOOK} 

\begin{enumerate}
    \item Systém bude umožňovat vyučujícím vkládat studijní materiály.
        \begin{itemize}
            \item Každý přihlášený uživatel s uživatelskou rolí Vyučující bude moci přidat pomocí jednoduchého formuláře nový PDF dokument. 
            \item Při vkládání PDF dokumentu do systému bude možné nastavit PDF dokumentu základní parametry.
            \item Dokument zůstane uložen v aplikaci s vygenerovanou adresou přes kterou bude možné dokument stáhnout. 
        \end{itemize}

    \item Systém bude umožňovat vyučujícím upravovat parametry vložených dokumentů.
        \begin{itemize}
            \item Každý přihlášený uživatel s uživatelskou rolí Vyučující bude mít možnost upravovat parametry dokumentů, které vložil do systému.
        \end{itemize}

    \item Systém bude umožňovat vyučujícím měnit a mazat dříve nahrané PDF dokumenty.
        \begin{itemize}
            \item Každý přihlášený uživatel s uživatelskou rolí Vyučující bude mít možnost zaměnit dříve nahraný PDF dokument za novější verzi.
            \item Další možností pro uživatele s uživatelskou rolí Vyučující bude možnost odstranění dříve vloženého PDF dokumentu ze systému. Odkazy směřující na smazaný dokument budou nefunkční.
        \end{itemize}

    \item Systém bude umožňovat prodej PDF dokumentů.
        \begin{itemize} 
            \item Každý přihlášený uživatel s uživatelskou rolí Vyučující bude mít možnost nastavit aplikaci tak, aby umožnila stažení PDF dokumentu až po provedení platby.
        \end{itemize}

    \item Systém bude umožňovat studentům stahovat PDF dokumety.
        \begin{itemize}
            \item Každý student, který vstoupí do aplikace přes odkaz na stažení PDF dokumentu a přihlásí se do systému bude mít možnost tento dokument stáhnout.
            \item Stažený PDF dokument bude označen vodoznakem obsahujícím informace o tom, kdy byl dokument stažen a také o tom, kým byl stažen.
        \end{itemize}

    \item Systém bude umožňovat administrátorovi prováďět jakékoliv úpravy v systému.
        \begin{itemize}
            \item Přihlášený uživatel s uživatelskou rolí Administrátor bude mít možnost měnit nastavení všech PDF dokumentů vložených do systému včetně jejich smazání.
        \end{itemize}

\end{enumerate}

\subsection{Nefunkční požadavky}
Nefunční požadavky představují určitá omezení a vlastnosti, která musí daný systém splňovat.\cite{UMLBOOK}

\begin{enumerate}
    \item Systém bude přehledný a snadno ovladatelný.
        \begin{itemize} 
            \item Systém bude navržen tak, aby bylo co nejjednodušší ho ovládat.
            \item Studentům bude stačit pouze odkaz na stažení PDF dokumentu a přihlášení k FELid na to, aby se dokument stáhl.
            \item Žádné klikání na odkazy ke stažený. Stažení začne automaticky hned po přihlášení.
        \end{itemize}
    \item Systém bude vyvinut na platformě Ruby on Rails
        \begin{itemize}
            \item Systém bude napsaný v programovacím jazyku Ruby \footnote{http://www.ruby-lang.org} s použitím frameworku Ruby on Rails\footnote{http://www.rubyonrails.com}.
        \end{itemize}
\end{enumerate}

\section{Případy užití}
V této kapitole se nachází případy užití vycházející z funkčních požadavků na aplikaci. 

\begin{description}
    \item[] \textbf{UC01 Přihlášení uživatele}
        \begin{description}
            \item[Uživatelské role:] Student, Vyučující, Administrátor
            \item[Scénář:] Scénář začíná vstupem uživatele na stránku, která vyžaduje přihlášení. 
                \begin{enumerate}
                    \item Uživatel vstoupí na stránku vyžadující přihlášení.  
                    \item Systém přesměruje uživatele na web s formulářem pro přihlášení do FELid.
                    \item Uživatel vyplní jméno a heslo a přihlásí.
                    \item Systém po přihlášení přesměruje uživatele zpět na požadovanou stránku a vyhodnotí uživatelskou role.
                        \begin{description}
                            \item[Vyhovuje:] Uživatel je vpuštěn na požadovanou stránku.
                            \item[Nevyhovuje:] Uživatel je informován o nedostatečných právech a je přesměrován na úvodní stránku aplikace.
                        \end{description}
                \end{enumerate}
        \end{description}
\end{description}

\begin{description}
    \item[] \textbf{UC02 Vložení PDF dokumentu}
        \begin{description}
            \item[Uživatelské role:] Student, Vyučující, Administrátor
            \item[Scénář:] Scénář začíná přístupem přihlášeného uživatele s vyhovují rolí na stránku pro správu PDF dokumentů.  
                \begin{enumerate}
                    \item Uživatel klikne na tlačítko přidat nový PDF dokument.  
                    \item Systém zobrazí formulář pro vložení nového PDF dokumentu. 
                    \item Uživatel vyplní formulář, připojí požadovaný PDF dokument a odešle formulář. 
                    \item Systém vyhodnotí validitu vstupních dat.
                        \begin{description}
                            \item[Validní:] Dokument je vložen do systému a je mu vygenerována adresa na stažení. 
                            \item[Nevalidní:] Uživatel je přesměrován zpět na formulář s výpisem chyb. Dokument není uložen do systému. 
                        \end{description}
                \end{enumerate}
        \end{description}
\end{description}

\begin{description}
    \item[] \textbf{UC03 Stažení PDF dokumentu}
        \begin{description}
            \item[Uživatelské role:] Vyučující, Administrátor
            \item[Scénář:] Scénář začíná přístupem přihlášeného uživatele s vyhovují rolí na příslušný odkaz.  
                \begin{enumerate}
                    \item Uživatel zadá odkaz pro stažení PDF dokumentu..  
                    \item Systém vyhodnotí existenci PDF dokumetu pod tímto odkazem.
                        \begin{description}
                            \item[Existuje:] Dokument je označem patřičným vodoznakem a automaticky začíná stahování. 
                            \item[Neexistuje:] Uživatel je upozorněn na neexistenci PDF dokumentu. 
                        \end{description}
                \end{enumerate}
        \end{description}
\end{description}
