\chapter{Popis problému, specifikace cíle}

\section{Popis problému}
Tato práce si klade za cíl vytvořit portál, který umožní zabezpečeně distribuovat elektronické studijní materiály ve formátu PDF. Portál si bude muset poradit s těmito problémy:

\begin{itemize}
    \item Umožnit autorům vložit vytvořený PDF dokument do systému a nabídnout ho ke stažení.
    \item Zajistit, že se k PDF dokumentům dostanou pouze studenti ČVUT FEL.
    \item Umožnit studentům zdarma si stáhnout PDF dokumenty.
    \item Každý stažený PDF dokument označit vodoznakem s informacemi o uživateli, který daný dokument stáhl.
    \item Umožnit připomínkovat jednotlivé stránky či části pomocí diskuzních příspěvků. 
    \item Umožnit vybrané PDF dokumenty stáhnout až po zaplacení předem stanovené finanční částky.
\end{itemize}

Tato bakalářská práce rozšiřuje již existující polofunkční prototyp vyvinut na fakultě v rámci studentských prací. Tento prototyp, umožňuje vložit do PDF dokumentů vodoznak s informacemi na základě vyplnění formuláře. Zároveň je možné tyto dokumenty volně stahovat pomocí vygenerovaného obsahu.  


\section{Cíle práce}
Cílem této práce je dokončit výše zmíněný prototyp tak, aby dokázal vyřešit uvedené problémy. Konkrétně se jedná o implementaci komponent.

\begin{itemize}
    \item FELid pro autentizaci uživatelů v rámci sítě ČVUT FEL.
    \item Upravit tvorbu vodoznaku tak, aby obsahoval informace o uživateli získané ze systému FELid\cite{FELID}.
    \item Umožnit připomínkovat jednotlivé stránky či části pomocí diskuzních příspěvků.
    \item Analyzovat a následně implementovat možnost rozšíření portálu o možnost komerčního prodeje PDF dokumentů.
\end{itemize}

\section{Rešerše}
S ohledem na specifičnost zadání práce je těžké najít existující projekty, které by byly alespoň z části podobné. Přesto zde může zmíněno pár projektů zabývajících se distribucí elektronických materiálů (většinou knih).  

\subsection{Amazon}
Amazon\cite{AMAZON} je asi nejznámější prodejce elektronických knih ne světě. Ačkoliv elektronické knihy nejsou jediným obchodním artiklem Amazonu stal se Amazon známým hlavně prodejem knih jak tradičních, tak i elektronických. Se svojí nabídkou elektronických knih se Amazon řadí mezi největší distributory na světě. 

Na Amazonu se dá nalézt spousta studijních materiálů vhodných ke studiu na ČVUT FEL. Pro mnohé, ale může být překážkou absence překladu knih do českého jazyka. Další nevýhodou může být potřeba speciální čtečky pro čtení knih. Čtečky pro formáty používané Amazonem se dají sehnat zdarma pro všechny běžné operační systémy a to jak desktopové, tak i mobilní.

Samozřejmostí na webu Amazonu je kvalitní přehledná kategorizace knih a kvalitně vyřešené fulltextové vyhledávání knih. Knihy jsou lehce dohledatelné přes název, jméno autora, žánr atd. Další výhodou je hodnocení knih a komentáře od ostatních uživatelů.  

Amazon samožrejmě neposkytuje elektronické knihy zdarma. Prodává je. S tím je spojená nutnost registrace na webu a provádění elektronických platebních operací. 

\subsection{Google Books}
Google Books\cite{GOOGLEBOOKS} je počin společnosti Google, Inc. Google Books využívá nejsilnější stránky Google a to je fulltextové vyhledávání. Google Books prohledává kompletní texty knih, které Google naskenoval a převedl do elektronické podoby, kterou umístil do své databáze.

\subsection{Google Play Books}
Google Play\cite{GOOGLEPLAY} Books od společnosti Google, Inc. je aplikace pro distribuci elektronických knih. Hlavní výhodou Google Play je integrace s chytrými mobilními telefony, tablety i osobními počítači. Google Play Book udržuje knihy v cloudu a je možné k nim přistupovat z kteréhokoliv zařízení připojeného ke Google Play pod Google účtem.  

Google Play Books zatím není přístupný v České republice, ale ve světě se pišní obrovskou knihovnou čítající více než 4 miliony elektronických knih\footnote{Stav v březnu 2012 uváděný firmou Google}.

\subsection{Palmknihy}
Palmknihy.cz je lokální projekt v České republice, který nabízí více než 900 knih v češtině včetně učebnic a odborných knih. Knihy zakoupené na webu palmknihy.cz je možné číst na různých typech zařízení jako jsou čtečky, tablety, chytré mobilní telefony a osobní počítače. 

Zřejmě největší nevýhodou tohoto projektu je zatím malá nabídka knih. Knihy je možné zakoupit pomocí bankovního převodu nebo bankovní kartou. 
