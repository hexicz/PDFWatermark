%% History:
% Pavel Tvrdik (26.12.2004)
%  + initial version for PhD Report
%
% Daniel Sykora (27.01.2005)
%
% Michal Valenta (3.12.2008)
% rada zmen ve formatovani (diky M. Duškovi, J. Holubovi a J. Žďárkovi)
% sjednoceni zdrojoveho kodu pro anglickou, ceskou, bakalarskou a diplomovou praci

% One-page layout: (proof-)reading on display
%%%% \documentclass[11pt,oneside,a4paper]{book}
% Two-page layout: final printing
\documentclass[11pt,twoside,a4paper]{book}   
%=-=-=-=-=-=-=-=-=-=-=-=--=%
% The user of this template may find useful to have an alternative to these 
% officially suggested packages:
\usepackage[czech, english]{babel}
\usepackage[T1]{fontenc} % pouzije EC fonty 
% pripadne pisete-li cesky, pak lze zkusit take:
% \usepackage[OT1]{fontenc} 
\usepackage[utf8]{inputenc}
%=-=-=-=-=-=-=-=-=-=-=-=--=%
% In case of problems with PDF fonts, one may try to uncomment this line:
%\usepackage{lmodern}
%=-=-=-=-=-=-=-=-=-=-=-=--=%
%=-=-=-=-=-=-=-=-=-=-=-=--=%
% Depending on your particular TeX distribution and version of conversion tools 
% (dvips/dvipdf/ps2pdf), some (advanced | desperate) users may prefer to use 
% different settings.
% Please uncomment the following style and use your CSLaTeX (cslatex/pdfcslatex) 
% to process your work. Note however, this file is in UTF-8 and a conversion to 
% your native encoding may be required. Some settings below depend on babel 
% macros and should also be modified. See \selectlanguage \iflanguage.
%\usepackage{czech}  %%%%%\usepackage[T1]{czech} %%%%[IL2] [T1] [OT1]
%=-=-=-=-=-=-=-=-=-=-=-=--=%

%%%%%%%%%%%%%%%%%%%%%%%%%%%%%%%%%%%%%%%
% Styles required in your work follow %
%%%%%%%%%%%%%%%%%%%%%%%%%%%%%%%%%%%%%%%
\usepackage{graphicx}
%\usepackage{indentfirst} %1. odstavec jako v cestine.

\usepackage{k336_thesis_macros} % specialni makra pro formatovani DP a BP
 % muzete si vytvorit i sva vlastni v souboru k336_thesis_macros.sty
 % najdete  radu jednoduchych definic, ktere zde ani nejsou pouzity
 % napriklad: 
 % \newcommand{\bfig}{\begin{figure}\begin{center}}
 % \newcommand{\efig}{\end{center}\end{figure}}
 % umoznuje pouzit prikaz \bfig namisto \begin{figure}\begin{center} atd.


%%%%%%%%%%%%%%%%%%%%%%%%%%%%%%%%%%%%%
% Zvolte jednu z moznosti 
% Choose one of the following options
%%%%%%%%%%%%%%%%%%%%%%%%%%%%%%%%%%%%%
%\newcommand\TypeOfWork{Diplomová práce} \typeout{Diplomova prace}
% \newcommand\TypeOfWork{Master's Thesis}   \typeout{Master's Thesis} 
\newcommand\TypeOfWork{Bakalářská práce}  \typeout{Bakalarska prace}
% \newcommand\TypeOfWork{Bachelor's Project}  \typeout{Bachelor's Project}


%%%%%%%%%%%%%%%%%%%%%%%%%%%%%%%%%%%%%
% Zvolte jednu z moznosti 
% Choose one of the following options
%%%%%%%%%%%%%%%%%%%%%%%%%%%%%%%%%%%%%
% nabidky jsou z: http://www.fel.cvut.cz/cz/education/bk/prehled.html

%\newcommand\StudProgram{Elektrotechnika a informatika, dobíhající, Bakalářský}
%\newcommand\StudProgram{Elektrotechnika a informatika, dobíhající, Magisterský}
% \newcommand\StudProgram{Elektrotechnika a informatika, strukturovaný, Bakalářský}
% \newcommand\StudProgram{Elektrotechnika a informatika, strukturovaný, Navazující magisterský}
\newcommand\StudProgram{Softwarové technologie a management, Bakalářský}
% English study:
% \newcommand\StudProgram{Electrical Engineering and Information Technology}  % bachelor programe
% \newcommand\StudProgram{Electrical Engineering and Information Technology}  %master program


%%%%%%%%%%%%%%%%%%%%%%%%%%%%%%%%%%%%%
% Zvolte jednu z moznosti 
% Choose one of the following options
%%%%%%%%%%%%%%%%%%%%%%%%%%%%%%%%%%%%%
% nabidky jsou z: http://www.fel.cvut.cz/cz/education/bk/prehled.html

%\newcommand\StudBranch{Výpočetní technika}   % pro program EaI bak. (dobihajici i strukt.)
%\newcommand\StudBranch{Výpočetní technika}   % pro prgoram EaI mag. (dobihajici i strukt.)
%\newcommand\StudBranch{Softwarové inženýrství}            %pro STM
\newcommand\StudBranch{Web a multimedia}                  % pro STM
%\newcommand\StudBranch{Computer Engineering}              % bachelor programe
%\newcommand\StudBranch{Computer Science and Engineering}  % master programe


%%%%%%%%%%%%%%%%%%%%%%%%%%%%%%%%%%%%%%%%%%%%
% Vyplnte nazev prace, autora a vedouciho
% Set up Work Title, Author and Supervisor
%%%%%%%%%%%%%%%%%%%%%%%%%%%%%%%%%%%%%%%%%%%%

\newcommand\WorkTitle{Portál pro zabezpečenou distribuci PDF dokumentů}
\newcommand\FirstandFamilyName{Jan Hovorka}
\newcommand\Supervisor{Ing. Martin Komárek}


% Pouzijete-li pdflatex, tak je prijemne, kdyz bude mit vase prace
% funkcni odkazy i v pdf formatu
\usepackage[
pdftitle={\WorkTitle},
pdfauthor={\FirstandFamilyName},
bookmarks=true,
colorlinks=true,
breaklinks=true,
urlcolor=red,
citecolor=blue,
linkcolor=blue,
unicode=true,
]
{hyperref}



% Extension posted by Petr Dlouhy in order for better sources reference (\cite{} command) especially in Czech.
% April 2010
% See comment over \thebibliography command for details.

\usepackage[square, numbers]{natbib}             % sazba pouzite literatury
%\usepackage{url}
%\DeclareUrlCommand\url{\def\UrlLeft{<}\def\UrlRight{>}\urlstyle{tt}}  %rm/sf/tt
%\renewcommand{\emph}[1]{\textsl{#1}}    % melo by byt kurziva nebo sklonene,
\let\oldUrl\url
\renewcommand\url[1]{<\texttt{\oldUrl{#1}}>}




\begin{document}

%%%%%%%%%%%%%%%%%%%%%%%%%%%%%%%%%%%%%
% Zvolte jednu z moznosti 
% Choose one of the following options
%%%%%%%%%%%%%%%%%%%%%%%%%%%%%%%%%%%%%
\selectlanguage{czech}
%\selectlanguage{english} 

% prikaz \typeout vypise vyse uvedena nastaveni v prikazovem okne
% pro pohodlne ladeni prace


\iflanguage{czech}{
	 \typeout{************************************************}
	 \typeout{Zvoleny jazyk: cestina}
	 \typeout{Typ prace: \TypeOfWork}
	 \typeout{Studijni program: \StudProgram}
	 \typeout{Obor: \StudBranch}
	 \typeout{Jmeno: \FirstandFamilyName}
	 \typeout{Nazev prace: \WorkTitle}
	 \typeout{Vedouci prace: \Supervisor}
	 \typeout{***************************************************}
	 \newcommand\Department{Katedra počítačů}
	 \newcommand\Faculty{Fakulta elektrotechnická}
	 \newcommand\University{České vysoké učení technické v Praze}
	 \newcommand\labelSupervisor{Vedoucí práce}
	 \newcommand\labelStudProgram{Studijní program}
	 \newcommand\labelStudBranch{Obor}
}{
	 \typeout{************************************************}
	 \typeout{Language: english}
	 \typeout{Type of Work: \TypeOfWork}
	 \typeout{Study Program: \StudProgram}
	 \typeout{Study Branch: \StudBranch}
	 \typeout{Author: \FirstandFamilyName}
	 \typeout{Title: \WorkTitle}
	 \typeout{Supervisor: \Supervisor}
	 \typeout{***************************************************}
	 \newcommand\Department{Department of Computer Science and Engineering}
	 \newcommand\Faculty{Faculty of Electrical Engineering}
	 \newcommand\University{Czech Technical University in Prague}
	 \newcommand\labelSupervisor{Supervisor}
	 \newcommand\labelStudProgram{Study Programme} 
	 \newcommand\labelStudBranch{Field of Study}
}




%%%%%%%%%%%%%%%%%%%%%%%%%%    Poznamky ke kompletaci prace
% Nasledujici pasaz uzavrenou v {} ve sve praci samozrejme 
% zakomentujte nebo odstrante. 
% Ve vysledne svazane praci bude nahrazena skutecnym 
% oficialnim zadanim vasi prace.
%{
%\pagenumbering{roman} \cleardoublepage \thispagestyle{empty}
%\chapter*{Na tomto místě bude oficiální zadání vaší práce}
%\begin{itemize}
%\item Toto zadání je podepsané děkanem a vedoucím katedry,
%\item musíte si ho vyzvednout na studiijním oddělení Katedry počítačů na Karlově náměstí,
%\item v jedné odevzdané práci bude originál tohoto zadání (originál zůstává po obhajobě na katedře),
%\item ve druhé bude na stejném místě neověřená kopie tohoto dokumentu (tato se vám vrátí po obhajobě).
%\end{itemize}
%\newpage
%}
%
%%%%%%%%%%%%%%%%%%%%%%%%%%    Titulni stranka / Title page 

\coverpagestarts

%%%%%%%%%%%%%%%%%%%%%%%%%%%    Podekovani / Acknowledgements 

\acknowledgements
\noindent
Zde můžete napsat své poděkování, pokud chcete a máte komu děkovat.


%%%%%%%%%%%%%%%%%%%%%%%%%%%   Prohlaseni / Declaration 

\declaration{V Praze dne 15.\,5.\,2012}
%\declaration{In Kořenovice nad Bečvárkou on May 15, 2008}


%%%%%%%%%%%%%%%%%%%%%%%%%%%%    Abstract 
 
\abstractpage

This bachelor thesis deals with the design and implementation of portal for secure distribution of PDF documents. The portal allows teachers to upload PDF documents. These PDF documents are offered to students to download for free or after payment.  

The student proceed to the application using received link to PDF document which can be downloaded for free or after payment. Before download all documents are labeled withe watermark which contains the name of student and the date of download. 

% Prace v cestine musi krome abstraktu v anglictine obsahovat i
% abstrakt v cestine.
\vglue60mm

\noindent{\Huge \textbf{Abstrakt}}
\vskip 2.75\baselineskip

\noindent

Tato bakalářská práce se zabývá návrhem a implementací portálu pro zabezpečenou distribuci PDF dokumentů. Portál umožňuje vyučujícím vkládat do systému dokumenty ve formátu PDF. Tyto PDF dokumenty jsou nabídnuty studentům zdarma nebo po zaplacené ke stažení.

Student přistoupí do aplikace pomocí obdrženého odkazu na PDF dokument, který si po zaplacení nebo zdarma stáhne. Dokumenty jsou před stažením označeny vodoznakem, který obsahuje jméno studenta, který dokument stáhl a datum stažení.

\noindent

%%%%%%%%%%%%%%%%%%%%%%%%%%%%%%%%  Obsah / Table of Contents 

\tableofcontents


%%%%%%%%%%%%%%%%%%%%%%%%%%%%%%%  Seznam obrazku / List of Figures 

\listoffigures


%%%%%%%%%%%%%%%%%%%%%%%%%%%%%%%  Seznam tabulek / List of Tables

\listoftables


%**************************************************************

\mainbodystarts
% horizontalní mezera mezi dvema odstavci
%\parskip=12pt
%11.12.2008 parskip + tolerance
\normalfont
\parskip=0.2\baselineskip plus 0.2\baselineskip minus 0.1\baselineskip

% Odsazeni prvniho radku odstavce resi class book (neaplikuje se na prvni 
% odstavce kapitol, sekci, podsekci atd.) Viz usepackage{indentfirst}.
% Chcete-li selektivne zamezit odsazeni 1. radku nektereho odstavce,
% pouzijte prikaz \noindent.

%**************************************************************

% Pro snadnejsi praci s vetsimi texty je rozumne tyto rozdelit
% do samostatnych souboru nejlepe dle kapitol a tyto potom vkladat
% pomoci prikazu \include{jmeno_souboru.tex} nebo \include{jmeno_souboru}.
% Napr.:
% \include{1_uvod}
% \include{2_teorie}
% atd...

% 1. kapitola
\chapter{Úvod}
V této kapitole bude popsána motivace pro vznik portálu pro zabezpečenou distribuci PDF\footnote{Portable Document Format} dokumentů.

\section{Motivace}
V dnešní době, kdy je nedostatek kvalitních studijních materiálů, je potřeba podporovat vznik těchto materiálů. U vyučujících se projevuje nevole k tvorbě kvalitních elektronických studijních materiálů, a to především z důvodu téměř nemožné kontroly nad jejich distribucí. Potencionální autoři kvalitních studijních materiálů nechtějí investovat svůj volný čas do psaní s tím, že se jejich díla budou nekontrolovatelně šířít po celém Internetu. Tradiční metoda distribuce studijních materiálů, formou skriptm nemusí být v mnoha případech řešením. V některých oborech, zejména technických, se standardy mění tak rychle, že není dost dobře možné vydávat dostatečně rychle aktualizovaná skripta vyhovující současným trendům.

V okamžiku, kdy by měli vyučující možnost dostatečné kontroly nad šířením svých studijních materiálů, mohli by rozšířit spektrum kvalitních volně dostupných studijních materiálů ke svým kurzům. Tyto materiály by byly dostupné pouze studentům ČVUT ke kvalitní příprave ke studiu. Současný stav nutí studenty spoléhat se pouze na nedostatečné materiály ve formě prezentací vytvořených za účelem přednášek a cvičení. V těchto prezentacích se nevyskytují zdaleka všechny důležité informace potřebné k úspěšnému zakončení kurzu. Přesto, že by si studenti měli dělat z přednášek a cvičení vlastní poznámky ke studiu, může se stát, že se studenti z různých důvodů (např. nemoci) nemohou dostavit na přednášku a pořídit si důležité poznámky, bez kterých nebudou schopni kurz úspěšně dokončit.

Další možností motivace vyučujících k vytváření kvalitních studijních materiálů by byla možnost zpřístupnit tyto materiály studentům ČVUT za určitý finanční obnos. Pokud budou tyto kvalitní studijní materiály za rozumnou a pro studenty přijatelnou cenu lehce přístupné, sníží se potřeba studentů shánět nelegální, většinou i nekvalitní a nevhodné materiály z různých neověřených zdrojů.

Z výše uvedeného vyplývá, že vytvoření kvalitního portálu pro zabezpečenou distribuci elektronických studijních materiálů by mohlo zvýšit chuť vyučujících k tvorby kvalitních studijních materiálů. Zároveň by vznik takovéhoto portálu mohl zlepšit kvalitu přípravy studentů na úspěšné absolvování studia na vysoké škole.  

\section{Historie}
Tato bakalářská práce rozšiřuje již existující polofunkční prototyp vyvinut na fakultě v rámci studentských prací. Tento prototyp, umožňuje vložit do PDF dokumentů vodoznak s informacemi na základě vyplnění formuláře. Zároveň je možné tyto dokumenty volně stahovat pomocí vygenerovaného obsahu.

% 2. kapitola
\chapter{Popis problému, specifikace cíle}

\section{Popis problému}
Tato práce si klade za cíl vytvořit portál, který umožní zabezpečeně distribuovat elektronické studijní materiály ve formátu PDF. Portál si bude muset poradit s těmito problémy:

\begin{itemize}
    \item Umožnit autorům vložit vytvořený PDF dokument do systému a nabídnout ho ke stažení.
    \item Zajistit, že se k PDF dokumentům dostanou pouze studenti ČVUT FEL.
    \item Umožnit studentům zdarma si stáhnout PDF dokumenty.
    \item Každý stažený PDF dokument označit vodoznakem s informacemi o uživateli, který daný dokument stáhl.
    \item Umožnit připomínkovat jednotlivé stránky či části pomocí diskuzních příspěvků. 
    \item Umožnit vybrané PDF dokumenty stáhnout až po zaplacení předem stanovené finanční částky.
\end{itemize}

Tato bakalářská práce rozšiřuje již existující polofunkční prototyp vyvinut na fakultě v rámci studentských prací. Tento prototyp, umožňuje vložit do PDF dokumentů vodoznak s informacemi na základě vyplnění formuláře. Zároveň je možné tyto dokumenty volně stahovat pomocí vygenerovaného obsahu.  


\section{Cíle práce}
Cílem této práce je dokončit výše zmíněný prototyp tak, aby dokázal vyřešit uvedené problémy. Konkrétně se jedná o implementaci komponent.

\begin{itemize}
    \item FELid pro autentizaci uživatelů v rámci sítě ČVUT FEL.
    \item Upravit tvorbu vodoznaku tak, aby obsahoval informace o uživateli získané ze systému FELid\cite{FELID}.
    \item Umožnit připomínkovat jednotlivé stránky či části pomocí diskuzních příspěvků.
    \item Analyzovat a následně implementovat možnost rozšíření portálu o možnost komerčního prodeje PDF dokumentů.
\end{itemize}

\section{Rešerše}
S ohledem na specifičnost zadání práce je těžké najít existující projekty, které by byly alespoň z části podobné. Přesto zde může zmíněno pár projektů zabývajících se distribucí elektronických materiálů (většinou knih).  

\subsection{Amazon}
Amazon\cite{AMAZON} je asi nejznámější prodejce elektronických knih ne světě. Ačkoliv elektronické knihy nejsou jediným obchodním artiklem Amazonu stal se Amazon známým hlavně prodejem knih jak tradičních, tak i elektronických. Se svojí nabídkou elektronických knih se Amazon řadí mezi největší distributory na světě. 

Na Amazonu se dá nalézt spousta studijních materiálů vhodných ke studiu na ČVUT FEL. Pro mnohé, ale může být překážkou absence překladu knih do českého jazyka. Další nevýhodou může být potřeba speciální čtečky pro čtení knih. Čtečky pro formáty používané Amazonem se dají sehnat zdarma pro všechny běžné operační systémy a to jak desktopové, tak i mobilní.

Samozřejmostí na webu Amazonu je kvalitní přehledná kategorizace knih a kvalitně vyřešené fulltextové vyhledávání knih. Knihy jsou lehce dohledatelné přes název, jméno autora, žánr atd. Další výhodou je hodnocení knih a komentáře od ostatních uživatelů.  

Amazon samožrejmě neposkytuje elektronické knihy zdarma. Prodává je. S tím je spojená nutnost registrace na webu a provádění elektronických platebních operací. 

\subsection{Google Books}
Google Books\cite{GOOGLEBOOKS} je počin společnosti Google, Inc. Google Books využívá nejsilnější stránky Google a to je fulltextové vyhledávání. Google Books prohledává kompletní texty knih, které Google naskenoval a převedl do elektronické podoby, kterou umístil do své databáze.

\subsection{Google Play Books}
Google Play\cite{GOOGLEPLAY} Books od společnosti Google, Inc. je aplikace pro distribuci elektronických knih. Hlavní výhodou Google Play je integrace s chytrými mobilními telefony, tablety i osobními počítači. Google Play Book udržuje knihy v cloudu a je možné k nim přistupovat z kteréhokoliv zařízení připojeného ke Google Play pod Google účtem.  

Google Play Books zatím není přístupný v České republice, ale ve světě se pišní obrovskou knihovnou čítající více než 4 miliony elektronických knih\footnote{Stav v březnu 2012 uváděný firmou Google}.

\subsection{Palmknihy}
Palmknihy.cz je lokální projekt v České republice, který nabízí více než 900 knih v češtině včetně učebnic a odborných knih. Knihy zakoupené na webu palmknihy.cz je možné číst na různých typech zařízení jako jsou čtečky, tablety, chytré mobilní telefony a osobní počítače. 

Zřejmě největší nevýhodou tohoto projektu je zatím malá nabídka knih. Knihy je možné zakoupit pomocí bankovního převodu nebo bankovní kartou. 

% 3. kapitola
\chapter{Analýza}

\section{Uživatelé}
Předpokládá se, že s aplikací budou pracovat různí uživatelé z akademického prostředí ČVUT FEL. Činnosti těchto uživatelů půjdou rozdělit do dvou hlavních skupin. 

První skupinou budou uživatelé, kteří budou aplikaci zásobovat obsahem. Tedy hlavně vyučující a další zaměstnanci školy. Jejich primární činností bude nahrát vybraný PDF dokument do aplikace, nastavit jeho vlastnosti a sdílet ho pomocí systémem vygenerovaných odkazů. Mezi další činnosti bude patřit editace vlastností PDF dokumentů, případně nahrazení dokumentu aktuální verzí a v neposlední řadě také mazání dokumentů.

Nejpočetnější skupina uživatelů aplikace bude tvořena studenty, kteří budou na základě získaných odkazů stahovat požadované materiály ve formě PDF dokumentů. 

\subsection{Uživatelské role}
Na základě analýzy uživatelů systému byly navrhnuty uživatelské role. Pro jednoduchost jsou role navrženy tak, aby je bylo možné generalizovat. Generalizace uživatelských rolí znamená, že uživatel s rolí vyššího stupně má zároveň vlastnosti uživatelů s rolemi nižších stupňů. Uživatelské role byly zvoleny takto:

\begin{itemize}
    \item \textbf{Host} je jakýkoliv uživatel, který vstoupí do aplikace a nepřihlásí se. Takový uživatel nemá prakticky žádné práva v aplikaci a bude mu zobrazena pouze úvodní stránka. Při pokusu vykonání jakékoliv aktivity v systému bude vyzván k přihlášení přes FELid.
    \item \textbf{Student} je přihlášený uživatel, který je v systému FELid evidován jako student. Student má právo stáhnout si pomocí odkazu PDF dokument, který bude před stažením označen vodotiskem se jménem studenta a datem stažení dokumentu.  
    \item \textbf{Vyučující} je také přihlášený uživatel, který je v systému FELid evidován jako zaměstnanec fakulty. Vyučující má stejné právo stahovat PDF dokumenty jako Student a zároveň může do aplikace vkládat a následně i spravovat nové dokumenty. Každý uživatel s rolí Vyučující bude mít právo spravovat pouze PDF dokumenty, které sám do systému vložil.
    \item \textbf{Administrátor} bude přihlášený uživatel, který bude mít právo spravovat aplikaci. Administrátor bude mít stejná prává jako Vyučují s tím, že narozdíl od něj bude mít přístup ke správě všech dokumentů od všech možných Vyučujících.
\end{itemize}

%\section{Doménový model}
%Doménový model bude velmi jednoduchý. Bude obsahovat jen několik málo entit. V této sekci jsou popsány použité entity a jejich atributy.
%
%\begin{itemize}
%    \item 
%\end{itemize}

\section{Požadavky}
V této sekci jsou uvedeny požadavky na portál z pohledu zadání práce.

\subsection{Funkční požadavky}
Funkční požadavky nám říkají, jaké akce bude systém umožňovat uživatelům provádět. Tyto akce jsou pokryty příůady užití. \cite{UMLBOOK} 

\begin{enumerate}
    \item Systém bude umožňovat vyučujícím vkládat studijní materiály.
        \begin{itemize}
            \item Každý přihlášený uživatel s uživatelskou rolí Vyučující bude moci přidat pomocí jednoduchého formuláře nový PDF dokument. 
            \item Při vkládání PDF dokumentu do systému bude možné nastavit PDF dokumentu základní parametry.
            \item Dokument zůstane uložen v aplikaci s vygenerovanou adresou přes kterou bude možné dokument stáhnout. 
        \end{itemize}

    \item Systém bude umožňovat vyučujícím upravovat parametry vložených dokumentů.
        \begin{itemize}
            \item Každý přihlášený uživatel s uživatelskou rolí Vyučující bude mít možnost upravovat parametry dokumentů, které vložil do systému.
        \end{itemize}

    \item Systém bude umožňovat vyučujícím měnit a mazat dříve nahrané PDF dokumenty.
        \begin{itemize}
            \item Každý přihlášený uživatel s uživatelskou rolí Vyučující bude mít možnost zaměnit dříve nahraný PDF dokument za novější verzi.
            \item Další možností pro uživatele s uživatelskou rolí Vyučující bude možnost odstranění dříve vloženého PDF dokumentu ze systému. Odkazy směřující na smazaný dokument budou nefunkční.
        \end{itemize}

    \item Systém bude umožňovat prodej PDF dokumentů.
        \begin{itemize} 
            \item Každý přihlášený uživatel s uživatelskou rolí Vyučující bude mít možnost nastavit aplikaci tak, aby umožnila stažení PDF dokumentu až po provedení platby.
        \end{itemize}

    \item Systém bude umožňovat studentům stahovat PDF dokumety.
        \begin{itemize}
            \item Každý student, který vstoupí do aplikace přes odkaz na stažení PDF dokumentu a přihlásí se do systému bude mít možnost tento dokument stáhnout.
            \item Stažený PDF dokument bude označen vodoznakem obsahujícím informace o tom, kdy byl dokument stažen a také o tom, kým byl stažen.
        \end{itemize}

    \item Systém bude umožňovat administrátorovi prováďět jakékoliv úpravy v systému.
        \begin{itemize}
            \item Přihlášený uživatel s uživatelskou rolí Administrátor bude mít možnost měnit nastavení všech PDF dokumentů vložených do systému včetně jejich smazání.
        \end{itemize}

\end{enumerate}

\subsection{Nefunkční požadavky}
Nefunční požadavky představují určitá omezení a vlastnosti, která musí daný systém splňovat.\cite{UMLBOOK}

\begin{enumerate}
    \item Systém bude přehledný a snadno ovladatelný.
        \begin{itemize} 
            \item Systém bude navržen tak, aby bylo co nejjednodušší ho ovládat.
            \item Studentům bude stačit pouze odkaz na stažení PDF dokumentu a přihlášení k FELid na to, aby se dokument stáhl.
            \item Žádné klikání na odkazy ke stažený. Stažení začne automaticky hned po přihlášení.
        \end{itemize}
    \item Systém bude vyvinut na platformě Ruby on Rails
        \begin{itemize}
            \item Systém bude napsaný v programovacím jazyku Ruby \footnote{http://www.ruby-lang.org} s použitím frameworku Ruby on Rails\footnote{http://www.rubyonrails.com}.
        \end{itemize}
\end{enumerate}

\section{Případy užití}
V této kapitole se nachází případy užití vycházející z funkčních požadavků na aplikaci. 

\begin{description}
    \item[] \textbf{UC01 Přihlášení uživatele}
        \begin{description}
            \item[Uživatelské role:] Student, Vyučující, Administrátor
            \item[Scénář:] Scénář začíná vstupem uživatele na stránku, která vyžaduje přihlášení. 
                \begin{enumerate}
                    \item Uživatel vstoupí na stránku vyžadující přihlášení.  
                    \item Systém přesměruje uživatele na web s formulářem pro přihlášení do FELid.
                    \item Uživatel vyplní jméno a heslo a přihlásí.
                    \item Systém po přihlášení přesměruje uživatele zpět na požadovanou stránku a vyhodnotí uživatelskou role.
                        \begin{description}
                            \item[Vyhovuje:] Uživatel je vpuštěn na požadovanou stránku.
                            \item[Nevyhovuje:] Uživatel je informován o nedostatečných právech a je přesměrován na úvodní stránku aplikace.
                        \end{description}
                \end{enumerate}
        \end{description}
\end{description}

\begin{description}
    \item[] \textbf{UC02 Vložení PDF dokumentu}
        \begin{description}
            \item[Uživatelské role:] Student, Vyučující, Administrátor
            \item[Scénář:] Scénář začíná přístupem přihlášeného uživatele s vyhovují rolí na stránku pro správu PDF dokumentů.  
                \begin{enumerate}
                    \item Uživatel klikne na tlačítko přidat nový PDF dokument.  
                    \item Systém zobrazí formulář pro vložení nového PDF dokumentu. 
                    \item Uživatel vyplní formulář, připojí požadovaný PDF dokument a odešle formulář. 
                    \item Systém vyhodnotí validitu vstupních dat.
                        \begin{description}
                            \item[Validní:] Dokument je vložen do systému a je mu vygenerována adresa na stažení. 
                            \item[Nevalidní:] Uživatel je přesměrován zpět na formulář s výpisem chyb. Dokument není uložen do systému. 
                        \end{description}
                \end{enumerate}
        \end{description}
\end{description}

\begin{description}
    \item[] \textbf{UC03 Stažení PDF dokumentu}
        \begin{description}
            \item[Uživatelské role:] Vyučující, Administrátor
            \item[Scénář:] Scénář začíná přístupem přihlášeného uživatele s vyhovují rolí na příslušný odkaz.  
                \begin{enumerate}
                    \item Uživatel zadá odkaz pro stažení PDF dokumentu..  
                    \item Systém vyhodnotí existenci PDF dokumetu pod tímto odkazem.
                        \begin{description}
                            \item[Existuje:] Dokument je označem patřičným vodoznakem a automaticky začíná stahování. 
                            \item[Neexistuje:] Uživatel je upozorněn na neexistenci PDF dokumentu. 
                        \end{description}
                \end{enumerate}
        \end{description}
\end{description}

% 4. kapitola
\chapter{Implementace}

\section{Technologie}
Použité technologie vychází z velké části z prototypu aplikace. Pro jednodušší rozšiřování aplikace jsou použité technologie zachovány.

\subsection{Server}
\subsection{Programovací jazyk a framework}
\subsection{Databáze}

% 5. kapitola
\chapter{Testování}

% 6. kapitola
\chapter{Závěr}

% Literatura
\bibliographystyle{csplainnat}
%\bibliographystyle{plain}
%\bibliographystyle{psc}
{
%JZ: 11.12.2008 Kdo chce mit v techto ukazkovych odkazech take odkaz na CSTeX:
\def\CS{$\cal C\kern-0.1667em\lower.5ex\hbox{$\cal S$}\kern-0.075em $}
\bibliography{ref}
}

%*****************************************************************************
\end{document}
